% -*- TeX:Rnw:UTF-8 -*-
% ----------------------------------------------------------------
% .R knitr file  ************************************************
% ----------------------------------------------------------------
%%
% \VignetteEngine{knitr::knitr}
% \VignetteIndexEntry{}
% \VignetteDepends{}
% \VignettePackage{}

\documentclass[a4paper,12pt]{article}\usepackage[]{graphicx}\usepackage[]{color}
% maxwidth is the original width if it is less than linewidth
% otherwise use linewidth (to make sure the graphics do not exceed the margin)
\makeatletter
\def\maxwidth{ %
  \ifdim\Gin@nat@width>\linewidth
    \linewidth
  \else
    \Gin@nat@width
  \fi
}
\makeatother

\usepackage{Sweave}


%\usepackage[slovene]{babel}
\usepackage[utf8]{inputenc} %% must be here for Sweave encoding check
\input{abpkg}
\input{abcmd}
\input{abpage}
\usepackage{pgf,pgfarrows,pgfnodes,pgfautomata,pgfheaps,pgfshade}
\usepackage{amsmath,amssymb}
\usepackage{colortbl}

\input{mysweave}


\setkeys{Gin}{width=0.8\textwidth}  % set graphicx parameter
\usepackage{lmodern}
\input{abfont}

% ----------------------------------------------------------------
\IfFileExists{upquote.sty}{\usepackage{upquote}}{}
\begin{document}
%% Sweave settings for includegraphics default plot size (Sweave default is 0.8)
%% notice this must be after begin{document}
%%% \setkeys{Gin}{width=0.9\textwidth}
% ----------------------------------------------------------------
\title{Development of package \pkg{pisar}}
\author{A. Blejec}
%\address{}%
%\email{}%
%
%\thanks{}%
%\subjclass{}%
%\keywords{}%

%\date{}%
%\dedicatory{}%
%\commby{}%
\maketitle
% ----------------------------------------------------------------
%\begin{abstract}
%
%\end{abstract}
% ----------------------------------------------------------------
\tableofcontents

\begin{Schunk}
\begin{Soutput}
Warning: package 'devtools' was built under R version 4.0.3
\end{Soutput}
\begin{Soutput}
Warning: package 'usethis' was built under R version 4.0.3
\end{Soutput}
\end{Schunk}

\clearpage
\section{Preparation}


Set package source directory

\begin{Schunk}
\begin{Sinput}
> purl <- TRUE
> (pkgPath <- gsub("/R/", "/R/!packages/", dirname(getwd())))
\end{Sinput}
\begin{Soutput}
[1] "C:/__D/OMIKE/pisar"
\end{Soutput}
\begin{Sinput}
> (pkgPath <- gsub("/R/", "/R/!packages/", dirname(getwd())))
\end{Sinput}
\begin{Soutput}
[1] "C:/__D/OMIKE/pisar"
\end{Soutput}
\begin{Sinput}
> (pkgName <- basename(dirname(getwd())))
\end{Sinput}
\begin{Soutput}
[1] "pisar"
\end{Soutput}
\begin{Sinput}
> dir(pkgPath)
\end{Sinput}
\begin{Soutput}
 [1] "_COPYRIGHT"                   "_DESCRIPTION"                
 [3] "_pkgdown.yml"                 "_VERSION"                    
 [5] "data"                         "DESCRIPTION"                 
 [7] "devel"                        "doc"                         
 [9] "docs"                         "FAIRDOM.log"                 
[11] "gitHeadInfo.gin"              "inst"                        
[13] "man"                          "NAMESPACE"                   
[15] "out"                          "pisar-initialize-corrected.R"
[17] "pisar.prj"                    "pisar.prj.bak"               
[19] "pisar.Rproj"                  "R"                           
[21] "README.md"                    "vignettes"                   
\end{Soutput}
\begin{Sinput}
> dir(file.path(pkgPath, "R"))
\end{Sinput}
\begin{Soutput}
[1] "pisar-initialize.R" "README.md"         
\end{Soutput}
\begin{Sinput}
> fctFile <- c("pisar-initialize.Rnw")
> (pkgFile <- gsub("\\.Rnw$", "\\.R", fctFile))
\end{Sinput}
\begin{Soutput}
[1] "pisar-initialize.R"
\end{Soutput}
\end{Schunk}

Files and directories:\\[12pt]
Original R source file(s): \url{pisar-initialize.Rnw}\\
Package directory: \url{C:/__D/OMIKE/pisar}\\
Package source file(s): \url{pisar-initialize.R}

%\section{Function definitions}
\clearpage
%<<child=fctFile,eval=!purl>>=
%@
\begin{Schunk}
\begin{Sinput}
> childtxt <- ""
\end{Sinput}
\end{Schunk}
\begin{Schunk}
\begin{Sinput}
> for (i in 1:length(fctFile)) {
+     childtxt <- paste(childtxt, knit_child(file.path("../devel", fctFile[i]), 
+         quiet = TRUE))
+ }
\end{Sinput}
\end{Schunk}

\clearpage
\section{Export as source for package}

perform steps for building a package. Extract code, make documentation files (\file{*.Rd}), build a package.

\subsection{Extract code}

Code is extracted to \url{pkgPath}.
%<<export to package,eval=purl>>=
%getwd()
%purl(file.path("../devel",fctFile),output=file.path(pkgPath,"R",pkgFile))
%@

\begin{Schunk}
\begin{Sinput}
> childtxt <- ""
> for (i in 1:length(fctFile)) {
+     cat(fctFile[i], "\n")
+     childtxt <- paste(childtxt, purl(file.path("../devel", fctFile[i]), 
+         output = file.path(pkgPath, "R", pkgFile[i])))
+ }
\end{Sinput}
\begin{Soutput}
pisar-initialize.Rnw 
\end{Soutput}
\begin{Soutput}


processing file: ../devel/pisar-initialize.Rnw
\end{Soutput}
\begin{Soutput}

  |                                                                  
  |                                                            |   0%
  |                                                                  
  |..                                                          |   3%
  |                                                                  
  |...                                                         |   5%
  |                                                                  
  |.....                                                       |   8%
  |                                                                  
  |......                                                      |  11%
  |                                                                  
  |........                                                    |  14%
  |                                                                  
  |..........                                                  |  16%
  |                                                                  
  |...........                                                 |  19%
  |                                                                  
  |.............                                               |  22%
  |                                                                  
  |...............                                             |  24%
  |                                                                  
  |................                                            |  27%
  |                                                                  
  |..................                                          |  30%
  |                                                                  
  |...................                                         |  32%
  |                                                                  
  |.....................                                       |  35%
  |                                                                  
  |.......................                                     |  38%
  |                                                                  
  |........................                                    |  41%
  |                                                                  
  |..........................                                  |  43%
  |                                                                  
  |............................                                |  46%
  |                                                                  
  |.............................                               |  49%
  |                                                                  
  |...............................                             |  51%
  |                                                                  
  |................................                            |  54%
  |                                                                  
  |..................................                          |  57%
  |                                                                  
  |....................................                        |  59%
  |                                                                  
  |.....................................                       |  62%
  |                                                                  
  |.......................................                     |  65%
  |                                                                  
  |.........................................                   |  68%
  |                                                                  
  |..........................................                  |  70%
  |                                                                  
  |............................................                |  73%
  |                                                                  
  |.............................................               |  76%
  |                                                                  
  |...............................................             |  78%
  |                                                                  
  |.................................................           |  81%
  |                                                                  
  |..................................................          |  84%
  |                                                                  
  |....................................................        |  86%
  |                                                                  
  |......................................................      |  89%
  |                                                                  
  |.......................................................     |  92%
  |                                                                  
  |.........................................................   |  95%
  |                                                                  
  |..........................................................  |  97%
  |                                                                  
  |............................................................| 100%
\end{Soutput}
\begin{Soutput}
output file: C:/__D/OMIKE/pisar/R/pisar-initialize.R
\end{Soutput}
\begin{Sinput}
> file.path(pkgPath, "R")
\end{Sinput}
\begin{Soutput}
[1] "C:/__D/OMIKE/pisar/R"
\end{Soutput}
\begin{Sinput}
> # dir(file.path(pkgPath,'R'))
\end{Sinput}
\end{Schunk}
 C:/__D/OMIKE/pisar/R/pisar-initialize.R

\clearpage
\subsection{Documentation}
Probably not needed if we do the check?
\begin{Schunk}
\begin{Sinput}
> devtools::document(pkgPath)
\end{Sinput}
\begin{Soutput}
Updating pisar documentation
\end{Soutput}
\begin{Soutput}
Loading pisar
\end{Soutput}
\begin{Soutput}
Writing NAMESPACE
Writing NAMESPACE
\end{Soutput}
\begin{Sinput}
> usethis::use_package("knitr")
\end{Sinput}
\begin{Soutput}
√ Setting active project to 'C:/__D/OMIKE/pisar'
\end{Soutput}
\begin{Soutput}
Warning in if (delta < 0) {: the condition has length > 1 and only the first element will be used
\end{Soutput}
\begin{Soutput}
Warning in if (delta > 0) {: the condition has length > 1 and only the first element will be used
\end{Soutput}
\begin{Soutput}
* Refer to functions with `knitr::fun()`
\end{Soutput}
\begin{Sinput}
> usethis::use_package("rio")
\end{Sinput}
\begin{Soutput}
* Refer to functions with `rio::fun()`
\end{Soutput}
\begin{Sinput}
> usethis::use_package("tools")
\end{Sinput}
\begin{Soutput}
* Refer to functions with `tools::fun()`
\end{Soutput}
\begin{Sinput}
> usethis::use_package("RCurl")
\end{Sinput}
\begin{Soutput}
* Refer to functions with `RCurl::fun()`
\end{Soutput}
\begin{Sinput}
> # usethis::use_package('httr') usethis::use_package('jsonlite')
> usethis::use_build_ignore(c("devel"))
\end{Sinput}
\end{Schunk}
\clearpage
\subsection{Check}
\begin{Schunk}
\begin{Sinput}
> # system.time(check <- devtools::check(pkgPath))
> system.time(miss <- devtools::missing_s3(pkgPath))
\end{Sinput}
\begin{Soutput}
Loading pisar
\end{Soutput}
\begin{Soutput}
   user  system elapsed 
   0.25    0.10    0.35 
\end{Soutput}
\end{Schunk}
\clearpage
\subsection{Check results}
\begin{Schunk}
\begin{Sinput}
> check()
\end{Sinput}
\begin{Soutput}
Updating pisar documentation
\end{Soutput}
\begin{Soutput}
Loading pisar
\end{Soutput}
\begin{Soutput}
Writing NAMESPACE
Writing NAMESPACE
-- Building --------------------------------------------------------------------------------------------------- pisar --
Setting env vars:
* CFLAGS    : -Wall -pedantic
* CXXFLAGS  : -Wall -pedantic
* CXX11FLAGS: -Wall -pedantic
------------------------------------------------------------------------------------------------------------------------
  
  
  
<U+221A>  checking for file 'C:\__D\OMIKE\pisar/DESCRIPTION'

  
  
  
-  preparing 'pisar':
   checking DESCRIPTION meta-information ...
  
   checking DESCRIPTION meta-information ... 
  
<U+221A>  checking DESCRIPTION meta-information

  
  
  
-  installing the package to build vignettes

  
  
  
   creating vignettes ...
  
   creating vignettes ... 
  
E  creating vignettes (1m 2.6s)
   duplicated vignette title:
     'my-vignette'
   
   --- re-building 'FAIRdom-and-R.Rmd' using rmarkdown
   Warning in engine$weave(file, quiet = quiet, encoding = enc) :
     Pandoc (>= 1.12.3) and/or pandoc-citeproc not available. Falling back to R Markdown v1.
   Warning in .Internal(as.vector(x, mode)) :
     closing unused connection 4 (https://testing.sysmo-db.org)
   Warning: closing unused connection 7 (https://testing.sysmo-db.org/assays/374)
   Warning: closing unused connection 6 (https://testing.sysmo-db.org/Studies/82)
   Warning: closing unused connection 5 (https://testing.sysmo-db.org/investigations/76)
   Warning: closing unused connection 4 (https://testing.sysmo-db.org/projects/54)
   --- finished re-building 'FAIRdom-and-R.Rmd'
   
   --- re-building 'FAIRdom2R.Rmd' using rmarkdown
   Warning in engine$weave(file, quiet = quiet, encoding = enc) :
     Pandoc (>= 1.12.3) and/or pandoc-citeproc not available. Falling back to R Markdown v1.
   --- finished re-building 'FAIRdom2R.Rmd'
   
   --- re-building 'my-vignette.Rmd' using rmarkdown
   Warning in engine$weave(file, quiet = quiet, encoding = enc) :
     Pandoc (>= 1.12.3) and/or pandoc-citeproc not available. Falling back to R Markdown v1.
   --- finished re-building 'my-vignette.Rmd'
   
   --- re-building 'HowTo-Use-pISA-tree-in-R.Rnw' using knitr
   Quitting from lines 215-217 (HowTo-Use-pISA-tree-in-R.Rnw) 
   Error: processing vignette 'HowTo-Use-pISA-tree-in-R.Rnw' failed with diagnostics:
   no applicable method for 'getMeta' applied to an object of class "c('pISAmeta', 'Dlist', 'data.frame')"
   --- failed re-building 'HowTo-Use-pISA-tree-in-R.Rnw'
   
   SUMMARY: processing the following file failed:
     'HowTo-Use-pISA-tree-in-R.Rnw'
   
   Error: Vignette re-building failed.
   Error: Duplicate vignette titles.
     Ensure that the %\VignetteIndexEntry lines in the vignette sources
     correspond to the vignette titles.
   Execution halted


\end{Soutput}
\begin{Soutput}
Error in (function (command = NULL, args = character(), error_on_status = TRUE, : System command 'Rcmd.exe' failed, exit status: 1, stdout + stderr:
E> * checking for file 'C:\__D\OMIKE\pisar/DESCRIPTION' ... OK
E> * preparing 'pisar':
E> * checking DESCRIPTION meta-information ... OK
E> * installing the package to build vignettes
E> * creating vignettes ... ERROR
E> duplicated vignette title:
E>   'my-vignette'
E> 
E> --- re-building 'FAIRdom-and-R.Rmd' using rmarkdown
E> Warning in engine$weave(file, quiet = quiet, encoding = enc) :
E>   Pandoc (>= 1.12.3) and/or pandoc-citeproc not available. Falling back to R Markdown v1.
E> Warning in .Internal(as.vector(x, mode)) :
E>   closing unused connection 4 (https://testing.sysmo-db.org)
E> Warning: closing unused connection 7 (https://testing.sysmo-db.org/assays/374)
E> Warning: closing unused connection 6 (https://testing.sysmo-db.org/Studies/82)
E> Warning: closing unused connection 5 (https://testing.sysmo-db.org/investigations/76)
E> Warning: closing unused connection 4 (https://testing.sysmo-db.org/projects/54)
E> --- finished re-building 'FAIRdom-and-R.Rmd'
E> 
E> --- re-building 'FAIRdom2R.Rmd' using rmarkdown
E> Warning in engine$weave(file, quiet = quiet, encoding = enc) :
E>   Pandoc (>= 1.12.3) and/or pandoc-citeproc not available. Falling back to R Markdown v1.
E> --- finished re-building 'FAIRdom2R.Rmd'
E> 
E> --- re-building 'my-vignette.Rmd' using rmarkdown
E> Warning in engine$weave(file, quiet = quiet, encoding = enc) :
E>   Pandoc (>= 1.12.3) and/or pandoc-citeproc not available. Falling back to R Markdown v1.
E> --- finished re-building 'my-vignette.Rmd'
E> 
E> --- re-building 'HowTo-Use-pISA-tree-in-R.Rnw' using knitr
E> Quitting from lines 215-217 (HowTo-Use-pISA-tree-in-R.Rnw) 
E> Error: processing vignette 'HowTo-Use-pISA-tree-in-R.Rnw' failed with diagnostics:
E> no applicable method for 'getMeta' applied to an object of class "c('pISAmeta', 'Dlist', 'data.frame')"
E> --- failed re-building 'HowTo-Use-pISA-tree-in-R.Rnw'
E> 
E> SUMMARY: processing the following file failed:
E>   'HowTo-Use-pISA-tree-in-R.Rnw'
E> 
E> Error: Vignette re-building failed.
E> Error: Duplicate vignette titles.
E>   Ensure that the %\VignetteIndexEntry lines in the vignette sources
E>   correspond to the vignette titles.
E> Execution halted
\end{Soutput}
\end{Schunk}
\clearpage
\subsection{Build a package}

Build the package

\begin{Schunk}
\begin{Sinput}
> devtools::build(pkgPath, manual = TRUE, quiet = FALSE)
\end{Sinput}
\begin{Soutput}
  
  
  
   checking for file 'C:\__D\OMIKE\pisar/DESCRIPTION' ...
  
<U+221A>  checking for file 'C:\__D\OMIKE\pisar/DESCRIPTION'

  
  
  
-  preparing 'pisar':
   checking DESCRIPTION meta-information ...
  
   checking DESCRIPTION meta-information ... 
  
<U+221A>  checking DESCRIPTION meta-information

  
  
  
-  installing the package to build vignettes

  
  
  
   creating vignettes ...
  
   creating vignettes ... 
  
E  creating vignettes (59.8s)
   duplicated vignette title:
     'my-vignette'
   
   --- re-building 'FAIRdom-and-R.Rmd' using rmarkdown
   Warning in engine$weave(file, quiet = quiet, encoding = enc) :
     Pandoc (>= 1.12.3) and/or pandoc-citeproc not available. Falling back to R Markdown v1.
   Warning in .Internal(as.vector(x, mode)) :
     closing unused connection 4 (https://testing.sysmo-db.org)
   Warning: closing unused connection 7 (https://testing.sysmo-db.org/assays/374)
   Warning: closing unused connection 6 (https://testing.sysmo-db.org/Studies/82)
   Warning: closing unused connection 5 (https://testing.sysmo-db.org/investigations/76)
   Warning: closing unused connection 4 (https://testing.sysmo-db.org/projects/54)
   --- finished re-building 'FAIRdom-and-R.Rmd'
   
   --- re-building 'FAIRdom2R.Rmd' using rmarkdown
   Warning in engine$weave(file, quiet = quiet, encoding = enc) :
     Pandoc (>= 1.12.3) and/or pandoc-citeproc not available. Falling back to R Markdown v1.
   --- finished re-building 'FAIRdom2R.Rmd'
   
   --- re-building 'my-vignette.Rmd' using rmarkdown
   Warning in engine$weave(file, quiet = quiet, encoding = enc) :
     Pandoc (>= 1.12.3) and/or pandoc-citeproc not available. Falling back to R Markdown v1.
   --- finished re-building 'my-vignette.Rmd'
   
   --- re-building 'HowTo-Use-pISA-tree-in-R.Rnw' using knitr
   Quitting from lines 215-217 (HowTo-Use-pISA-tree-in-R.Rnw) 
   Error: processing vignette 'HowTo-Use-pISA-tree-in-R.Rnw' failed with diagnostics:
   no applicable method for 'getMeta' applied to an object of class "c('pISAmeta', 'Dlist', 'data.frame')"
   --- failed re-building 'HowTo-Use-pISA-tree-in-R.Rnw'
   
   SUMMARY: processing the following file failed:
     'HowTo-Use-pISA-tree-in-R.Rnw'
   
   Error: Vignette re-building failed.
   Error: Duplicate vignette titles.
     Ensure that the %\VignetteIndexEntry lines in the vignette sources
     correspond to the vignette titles.
   Execution halted


\end{Soutput}
\begin{Soutput}
Error in (function (command = NULL, args = character(), error_on_status = TRUE, : System command 'Rcmd.exe' failed, exit status: 1, stdout + stderr:
E> * checking for file 'C:\__D\OMIKE\pisar/DESCRIPTION' ... OK
E> * preparing 'pisar':
E> * checking DESCRIPTION meta-information ... OK
E> * installing the package to build vignettes
E> * creating vignettes ... ERROR
E> duplicated vignette title:
E>   'my-vignette'
E> 
E> --- re-building 'FAIRdom-and-R.Rmd' using rmarkdown
E> Warning in engine$weave(file, quiet = quiet, encoding = enc) :
E>   Pandoc (>= 1.12.3) and/or pandoc-citeproc not available. Falling back to R Markdown v1.
E> Warning in .Internal(as.vector(x, mode)) :
E>   closing unused connection 4 (https://testing.sysmo-db.org)
E> Warning: closing unused connection 7 (https://testing.sysmo-db.org/assays/374)
E> Warning: closing unused connection 6 (https://testing.sysmo-db.org/Studies/82)
E> Warning: closing unused connection 5 (https://testing.sysmo-db.org/investigations/76)
E> Warning: closing unused connection 4 (https://testing.sysmo-db.org/projects/54)
E> --- finished re-building 'FAIRdom-and-R.Rmd'
E> 
E> --- re-building 'FAIRdom2R.Rmd' using rmarkdown
E> Warning in engine$weave(file, quiet = quiet, encoding = enc) :
E>   Pandoc (>= 1.12.3) and/or pandoc-citeproc not available. Falling back to R Markdown v1.
E> --- finished re-building 'FAIRdom2R.Rmd'
E> 
E> --- re-building 'my-vignette.Rmd' using rmarkdown
E> Warning in engine$weave(file, quiet = quiet, encoding = enc) :
E>   Pandoc (>= 1.12.3) and/or pandoc-citeproc not available. Falling back to R Markdown v1.
E> --- finished re-building 'my-vignette.Rmd'
E> 
E> --- re-building 'HowTo-Use-pISA-tree-in-R.Rnw' using knitr
E> Quitting from lines 215-217 (HowTo-Use-pISA-tree-in-R.Rnw) 
E> Error: processing vignette 'HowTo-Use-pISA-tree-in-R.Rnw' failed with diagnostics:
E> no applicable method for 'getMeta' applied to an object of class "c('pISAmeta', 'Dlist', 'data.frame')"
E> --- failed re-building 'HowTo-Use-pISA-tree-in-R.Rnw'
E> 
E> SUMMARY: processing the following file failed:
E>   'HowTo-Use-pISA-tree-in-R.Rnw'
E> 
E> Error: Vignette re-building failed.
E> Error: Duplicate vignette titles.
E>   Ensure that the %\VignetteIndexEntry lines in the vignette sources
E>   correspond to the vignette titles.
E> Execution halted
\end{Soutput}
\begin{Sinput}
> devtools::load_all()
\end{Sinput}
\begin{Soutput}
Loading pisar
\end{Soutput}
\end{Schunk}



Install

\begin{Schunk}
\begin{Sinput}
> devtools::install(pkgPath)
\end{Sinput}
\begin{Soutput}
vctrs      (0.3.4  -> 0.3.6 ) [CRAN]
fansi      (0.4.1  -> 0.4.2 ) [CRAN]
crayon     (1.3.4  -> 1.4.0 ) [CRAN]
cli        (2.2.0  -> 2.3.0 ) [CRAN]
Rcpp       (1.0.5  -> 1.0.6 ) [CRAN]
tibble     (3.0.4  -> 3.0.6 ) [CRAN]
cpp11      (0.2.5  -> 0.2.6 ) [CRAN]
forcats    (0.5.0  -> 0.5.1 ) [CRAN]
xfun       (0.19   -> 0.20  ) [CRAN]
data.table (1.13.2 -> 1.13.6) [CRAN]
knitr      (1.30   -> 1.31  ) [CRAN]
\end{Soutput}
\begin{Soutput}
Installing 11 packages: vctrs, fansi, crayon, cli, Rcpp, tibble, cpp11, forcats, xfun, data.table, knitr
\end{Soutput}
\begin{Soutput}
Error: (converted from warning) package 'knitr' is in use and will not be installed
\end{Soutput}
\begin{Sinput}
> ## str(out)(shell(paste( file.path(R.home('bin'),'Rcmd.exe'), ' INSTALL
> ## --no-multiarch --with-keep.source', pkgPath) ,intern=FALSE))
\end{Sinput}
\end{Schunk}

Load
\begin{Schunk}
\begin{Sinput}
> devtools::load_all()
\end{Sinput}
\begin{Soutput}
Loading pisar
\end{Soutput}
\begin{Sinput}
> cat("Package:", pkgName, "\n")
\end{Sinput}
\begin{Soutput}
Package: pisar 
\end{Soutput}
\begin{Sinput}
> library(pkgName, character.only = TRUE)
> help(package = (pkgName))
\end{Sinput}
\begin{Soutput}
Warning in file.show(outFile, delete.file = TRUE, title = gettextf("Documentation for package %s", : '"C:\Program Files (x86)\EmEditor\EmEditor.exe"' not found
\end{Soutput}
\end{Schunk}

\section{PDF documentation}

\begin{Schunk}
\begin{Sinput}
> (pkgName <- basename(dirname(getwd())))
\end{Sinput}
\begin{Soutput}
[1] "pisar"
\end{Soutput}
\begin{Sinput}
> (instPath <- find.package(pkgName))
\end{Sinput}
\begin{Soutput}
[1] "C:/__D/OMIKE/pisar"
\end{Soutput}
\begin{Sinput}
> pdfFile <- file.path(getwd(), paste(pkgName, "pdf", sep = "."))
> if (file.exists(pdfFile)) file.remove(pdfFile)
\end{Sinput}
\begin{Soutput}
[1] TRUE
\end{Soutput}
\begin{Sinput}
> system(paste(shQuote(file.path(R.home("bin"), "R")), "CMD", "Rd2pdf", shQuote(instPath)))
\end{Sinput}
\begin{Soutput}
[1] 0
\end{Soutput}
\begin{Sinput}
> dir(pattern = pkgName)
\end{Sinput}
\begin{Soutput}
 [1] "pisar-api.Rnw"                  "pisar-functions.Rnw"           
 [3] "pisar-initialize.Rnw"           "pisar-initialize.synctex(busy)"
 [5] "pisar-initialize.tex"           "pisar-makePkg.log"             
 [7] "pisar-makePkg.pdf"              "pisar-makePkg.Rnw"             
 [9] "pisar-makePkg.synctex"          "pisar-makePkg.tex"             
\end{Soutput}
\end{Schunk}

\begin{Schunk}
\begin{Sinput}
> help(package = (pkgName), help_type = "pdf")
\end{Sinput}
\begin{Soutput}
Warning in file.show(outFile, delete.file = TRUE, title = gettextf("Documentation for package %s", : '"C:\Program Files (x86)\EmEditor\EmEditor.exe"' not found
\end{Soutput}
\end{Schunk}

Send package to R Windows builder

\begin{Schunk}
\begin{Sinput}
> devtools::build_win(pkgPath)
\end{Sinput}
\end{Schunk}

%<<child='encChild2.Rnw'>>=
%@
% ----------------------------------------------------------------
%\bibliographystyle{chicago}
%\addcontentsline{toc}{section}{\refname}
%\bibliography{ab-general}
%--------------------------------------------------------------

%\clearpage
%\appendix
%\phantomsection\addcontentsline{toc}{section}{\appendixname}
%\section{\R\ funkcije}
%\input{}

\clearpage
\section*{SessionInfo}
{\small
Windows 10 x64 (build 19041) 
\begin{itemize}\raggedright
  \item R version 4.0.2 (2020-06-22), \verb|x86_64-w64-mingw32|
  \item Locale: \verb|LC_COLLATE=Slovenian_Slovenia.1250|, \verb|LC_CTYPE=Slovenian_Slovenia.1250|, \verb|LC_MONETARY=Slovenian_Slovenia.1250|, \verb|LC_NUMERIC=C|, \verb|LC_TIME=Slovenian_Slovenia.1250|
  \item Running under: \verb|Windows 10 x64 (build 19041)|
  \item Matrix products: default
  \item Base packages: base, datasets, graphics, grDevices,
    methods, stats, utils
  \item Other packages: devtools~2.3.2, knitr~1.30,
    pisar~0.1.0.9000, usethis~2.0.0
  \item Loaded via a namespace (and not attached):
    assertthat~0.2.1, bitops~1.0-6, callr~3.5.1,
    cellranger~1.1.0, cli~2.2.0, compiler~4.0.2, crayon~1.3.4,
    curl~4.3, data.table~1.13.2, desc~1.2.0, digest~0.6.27,
    ellipsis~0.3.1, evaluate~0.14, fansi~0.4.1, forcats~0.5.0,
    foreign~0.8-80, formatR~1.7, fs~1.5.0, glue~1.4.2,
    haven~2.3.1, hms~1.0.0, httr~1.4.2, jsonlite~1.7.2,
    lifecycle~0.2.0, magrittr~2.0.1, memoise~1.1.0,
    openxlsx~4.2.3, pillar~1.4.7, pkgbuild~1.2.0,
    pkgconfig~2.0.3, pkgload~1.1.0, prettyunits~1.1.1,
    processx~3.4.5, ps~1.5.0, purrr~0.3.4, R6~2.5.0,
    Rcpp~1.0.5, RCurl~1.98-1.2, readxl~1.3.1, remotes~2.2.0,
    rio~0.5.16, rlang~0.4.10, roxygen2~7.1.1, rprojroot~2.0.2,
    rstudioapi~0.13, sessioninfo~1.1.1, stringi~1.5.3,
    stringr~1.4.0, testthat~3.0.1, tibble~3.0.4, tools~4.0.2,
    vctrs~0.3.4, withr~2.3.0, xfun~0.19, xml2~1.3.2, zip~2.1.1
\end{itemize}
Project path:\verb' C:/__D/OMIKE/pisar '\\
Main file :\verb' ../devel/pisar-makePkg.Rnw '


\subsection*{View as vignette}
Project files can be viewed by pasting this code to \R\ console:\\
\begin{Schunk}
\begin{Sinput}
> projectName <-"pisar";  mainFile <-"pisar-makePkg"

\end{Sinput}
\end{Schunk}
\begin{Schunk}
\begin{Sinput}
> commandArgs()
> library(tkWidgets)
> openPDF(file.path(dirname(getwd()),"doc",
> paste(mainFile,"PDF",sep=".")))
> viewVignette("viewVignette", projectName, #
> file.path("../devel",paste(mainFile,"Rnw",sep=".")))
> #
\end{Sinput}
\end{Schunk}

\vfill \hrule \vspace{3pt} \footnotesize{
%Revision \SVNId\hfill (c) A. Blejec%\input{../_COPYRIGHT.}
%\SVNRevision ~/~ \SVNDate
\noindent
\texttt{Git Revision: \gitCommitterUnixDate \gitAbbrevHash{} (\gitCommitterDate)} \hfill \copyright A. Blejec\\
\texttt{ \gitReferences} \hfill \verb'../devel/pisar-makePkg.Rnw'\\

}




\end{document}
% ----------------------------------------------------------------
