 %\VignetteEngine{knitr::knitr}
% -*- TeX:Rnw:UTF-8 -*-
% ----------------------------------------------------------------
% .R knitr file  ************************************************
% ----------------------------------------------------------------
%%
%\VignetteIndexEntry{HowTo Use pISA-tree in R}
%\VignetteDepends{}
%VignetteBuilder{knitr}
%\VignetteEngine{knitr::knit}
%\VignettePackage{pisar}
%\VignetteEncoding{UTF-8}

\documentclass[a4paper,12pt]{article}\usepackage[]{graphicx}\usepackage[]{color}
% maxwidth is the original width if it is less than linewidth
% otherwise use linewidth (to make sure the graphics do not exceed the margin)
\makeatletter
\def\maxwidth{ %
  \ifdim\Gin@nat@width>\linewidth
    \linewidth
  \else
    \Gin@nat@width
  \fi
}
\makeatother

\usepackage{Sweave}


%\usepackage[slovene]{babel}
\usepackage[utf8]{inputenc} %% must be here for Sweave encoding check
\providecommand\code{\bgroup\@codex}
\def\@codex#1{{\normalfont\ttfamily\hyphenchar\font=-1  #1}\egroup}
\providecommand{\codei}[1]{\code{#1}\index{\code{#1}}}
\providecommand{\NA}{\code{NA}}
\providecommand{\file}{\code}
\providecommand{\kbd}[1]{{\normalfont\texttt{#1}}}
\providecommand{\key}[1]{{\normalfont\texttt{\uppercase{#1}}}}
\providecommand{\var}[1]{{\normalfont\textsl{#1}}}
\providecommand{\fct}[1]{{\ttfamily\textbf{#1}()}}
\newcommand{\ii}[1]{{\it #1}} % for index page reference in italic
\providecommand{\fcti}[1]{\fct{#1}\index{#1@{\fct{#1}}}}
\providecommand{\strong}[1]{{\normalfont\fontseries{b}\selectfont  #1}}
\let\pkg=\strong
\providecommand{\email}[1]{\href{mailto:#1}{\normalfont\small\texttt{#1}}}

%%\input{abpkg}
%%\input{abcmd}
%%\input{abpage}
%%\usepackage{pgf,pgfarrows,pgfnodes,pgfautomata,pgfheaps,pgfshade}
%%\usepackage{amsmath,amssymb}
%%\usepackage{colortbl}
%%
%%\input{mysweave}


\setkeys{Gin}{width=0.8\textwidth}  % set graphicx parameter
\usepackage{lmodern}
%%\input{abfont}

% ----------------------------------------------------------------
\IfFileExists{upquote.sty}{\usepackage{upquote}}{}
\begin{document}
%% Sweave settings for includegraphics default plot size (Sweave default is 0.8)
%% notice this must be after begin{document}
%%% \setkeys{Gin}{width=0.9\textwidth}
% ----------------------------------------------------------------
\title{HowTo Use pISA-tree in R}
\author{Andrej Blejec\\
National Institute of Biology\\
Ljubljana, Slovenia\\
%\address{}%
\small{Email: \code{andrej.blejec@nib.si}}
}%
%
%\thanks{}%
%\subjclass{}%
%\keywords{}%

%\date{}%
%\dedicatory{}%
%\commby{}%
\maketitle
% ----------------------------------------------------------------
%\begin{abstract}
%
%\end{abstract}
% ----------------------------------------------------------------
\tableofcontents



\section{Introduction}
The package \pkg{pisar} implements functions that can be used to
use pISA-tree metadata and pISA-tree standardized directory tree in R.
The benefit for making reports reproducible is to get all details about the
file positions from the pISA metadata.

pISA-tree is a system to organize research data in a structured way.

\emph{Enter the pISA-tree description here}

\section{Get the pISA-tree information}

First you need to load the \pkg{pisar} package.

\begin{Schunk}
\begin{Sinput}
> library(pisar)
\end{Sinput}
\end{Schunk}

A data analysis report usually starts from a directory, that is
included in an Assay layer, with class \code{DRY} and type \code{R}.
The \pkg{pisar} package comes with a pISA directory tree that can be
used for demonstration:

\begin{Schunk}
\begin{Sinput}
> astring <- "_p_Demo/_I_Test/_S_Show/_A_Work-R/other"
\end{Sinput}
\end{Schunk}

We have a project \code{'Demo'}, an Investigation \code{Test} with Study \code{Show}  and Assay \code{Work}. In our case, we will use the folder \file{other} as R session working directory.

\begin{Schunk}
\begin{Sinput}
> .pisaPath <- system.file("extdata", astring, package = "pisar")
> oldwd <- getwd()
> if (interactive()) {
+     oldwd <- setwd(.pisaPath)
+     strsplit(getwd(), "/")
+ }
\end{Sinput}
\end{Schunk}

If you are working with knitr, use:
\begin{Schunk}
\begin{Sinput}
> opts_knit$set(root.dir = .pisaPath)
\end{Sinput}
\end{Schunk}

We can check the content of the working directory, that contains at least a \file{README.MD} file:
\begin{Schunk}
\begin{Sinput}
> dir()
\end{Sinput}
\begin{Soutput}
[1] "_outputFile.R" "README.MD"    
\end{Soutput}
\begin{Sinput}
> readLines("README.MD")
\end{Sinput}
\begin{Soutput}
[1] "# Other files for assay Work-R "
\end{Soutput}
\end{Schunk}

\section{Layer's root directory}

For navigation around the pISA tree, we can rely on layered structure. All layers are somewhere above the working directory. This enablesrelative paths: read two dots (..) as 'parent' and one dot (.) as 'here'.

Let's see the relative paths to the layers and check existence of the metadata files:

Project:
\begin{Schunk}
\begin{Sinput}
> .proot <- getRoot("p")
> .proot
\end{Sinput}
\begin{Soutput}
[1] "../../../.."
\end{Soutput}
\begin{Sinput}
> dir(.proot, pattern = glob2rx("*.TXT"))
\end{Sinput}
\begin{Soutput}
[1] "_PROJECT_METADATA.TXT"
\end{Soutput}
\end{Schunk}

Investigation:
\begin{Schunk}
\begin{Sinput}
> .iroot <- getRoot("I")
> .iroot
\end{Sinput}
\begin{Soutput}
[1] "../../.."
\end{Soutput}
\begin{Sinput}
> dir(.iroot, pattern = glob2rx("*.TXT"))
\end{Sinput}
\begin{Soutput}
[1] "_INVESTIGATION_METADATA.TXT"
\end{Soutput}
\end{Schunk}

Study:
\begin{Schunk}
\begin{Sinput}
> .sroot <- getRoot("S")
> .sroot
\end{Sinput}
\begin{Soutput}
[1] "../.."
\end{Soutput}
\begin{Sinput}
> dir(.sroot, pattern = glob2rx("*.TXT"))
\end{Sinput}
\begin{Soutput}
[1] "_STUDY_METADATA.TXT"
\end{Soutput}
\end{Schunk}

Assay:
\begin{Schunk}
\begin{Sinput}
> .aroot <- getRoot("A")
> .aroot
\end{Sinput}
\begin{Soutput}
[1] ".."
\end{Soutput}
\begin{Sinput}
> dir(.aroot, pattern = glob2rx("*.TXT"))
\end{Sinput}
\begin{Soutput}
[1] "_Assay_METADATA.TXT"
\end{Soutput}
\end{Schunk}

\section{Metadata files}

Metadata files contain Key/Value pair lines with detailed information about the layer. We can read them with the function \fct{readMeta}:

\begin{Schunk}
\begin{Sinput}
> .proot
\end{Sinput}
\begin{Soutput}
[1] "../../../.."
\end{Soutput}
\begin{Sinput}
> .pmeta <- readMeta(.proot)
\end{Sinput}
\end{Schunk}

In addition to read the metadata information as a \code{data.frame}, function
\fct{readMeta} sets two additional class values, which enables nicer printing:

\begin{Schunk}
\begin{Sinput}
> str(.pmeta)
\end{Sinput}
\begin{Soutput}
Classes 'pISAmeta', 'Dlist' and 'data.frame':	16 obs. of  2 variables:
 $ Key  : chr  "Short Name:" "Title:" "Description:" "pISA projects path:" ...
 $ Value: chr  "Demo" "Project demonstration" "This is a demo project for R package 'pisar'." "D:/OMIKE/pISA" ...
\end{Soutput}
\end{Schunk}

Indentation of the Value part depends on the line widths and we will set it to some higher value.

\begin{Schunk}
\begin{Sinput}
> .pmeta
\end{Sinput}
\begin{Soutput}
 Key                           Value
 ---                           -----
 Short Name:                   Demo
 Title:                        Project demonstration
 Description:                  This is a demo project for R package 'pisar'.
 pISA projects path:           D:/OMIKE/pISA
 Local pISA-tree organisation: NIB
 pISA project creation date:   2019-10-15
 pISA project creator:         AB
 Project funding code:         *
 Project coordinator:          *
 Project partners:             *
 Project start date:           2018-01-01
 Project end date:             2021-12-31
 Principal investigator:       *
 License:                      CC BY 4.0
 Sharing permission:           Private
 Upload to FAIRDOMHub:         Yes
\end{Soutput}
\end{Schunk}

We can get other metadata information:

\begin{Schunk}
\begin{Sinput}
> (.imeta <- readMeta(.iroot))
\end{Sinput}
\begin{Soutput}
 Key                               Value
 ---                              -----
 Short Name:                       Test
 Title:                            Test investigation
 Description:                      Investigation - for demonstration purposes.
 Phenodata:                        ./phenodata_20191015.txt
 pISA Investigation creation date: 2019-10-15
 pISA Investigation creator:       AB
 Principal investigator:           *
 License:                          CC BY 4.0
 Sharing permission:               Private
 Upload to FAIRDOMHub:             Yes
\end{Soutput}
\begin{Sinput}
> (.smeta <- readMeta(.sroot))
\end{Sinput}
\begin{Soutput}
 Key                       Value
 ---                       -----
 Short Name:               Show
 Title:                    Testing study
 Description:              Test study for demonstration only.
 Raw Data:                 
 pISA Study creation date: 2019-10-15
 pISA Study creator:       AB
 Principal investigator:   *
 License:                  CC BY 4.0
 Sharing permission:       Private
 Upload to FAIRDOMHub:     Yes
\end{Soutput}
\begin{Sinput}
> (.ameta <- readMeta(.aroot))
\end{Sinput}
\begin{Soutput}
 Key                       Value
 ---                       -----
 Short Name:               Work-R
 Assay Class:              DRY
 Assay Type:               R
 Title:                    Working in assay
 Description:              Not really working, just testing :)
 pISA Assay creation date: 2019-10-15
 pISA Assay creator:       AB
 Analyst:                  AB
 Phenodata:                ../../phenodata_20191015.txt
 Featuredata:              
 Data:                     
\end{Soutput}
\end{Schunk}

\section{Getting specific parts from the metadata object}

In addition to usual data extraction methods (using square brackets), we can do it with a function \fct{getMeta}:

Get the title of the project:

\begin{Schunk}
\begin{Sinput}
> getMeta(.pmeta, "Title")
\end{Sinput}
\begin{Soutput}
[1] "Project demonstration"
\end{Soutput}
\begin{Sinput}
> getMeta(.pmeta, "Title:")
\end{Sinput}
\begin{Soutput}
[1] "Project demonstration"
\end{Soutput}
\end{Schunk}

As we can see, the requested item value can be used with or without the colon.

\begin{Schunk}
\begin{Sinput}
> getMeta(.ameta, "Description")
\end{Sinput}
\begin{Soutput}
[1] "Not really working, just testing :)"
\end{Soutput}
\end{Schunk}

\section{All together}

All metadata information and some additional useful directory path strings can be extracted with the function \fct{pisa}:

\begin{Schunk}
\begin{Sinput}
> p <- pisa()
\end{Sinput}
\end{Schunk}

The result is a list with metadata information. The elements are described in the function's help file \code{`?pisa`}.

Access the individual parts is as usual:

\begin{Schunk}
\begin{Sinput}
> p$A
\end{Sinput}
\begin{Soutput}
$name
[1] "_A_Work-R"

$root
[1] ".."

$meta
 Key                       Value
 ---                       -----
 Short Name:               Work-R
 Assay Class:              DRY
 Assay Type:               R
 Title:                    Working in assay
 Description:              Not really working, just testing :)
 pISA Assay creation date: 2019-10-15
 pISA Assay creator:       AB
 Analyst:                  AB
 Phenodata:                ../../phenodata_20191015.txt
 Featuredata:              
 Data:                     
\end{Soutput}
\end{Schunk}

Path to Study level:

\begin{Schunk}
\begin{Sinput}
> p$S$root
\end{Sinput}
\begin{Soutput}
[1] "../.."
\end{Soutput}
\end{Schunk}

In addition, we get  dot-named objects (similar as above) in the global environment. For details, see help for function \fct{pisa}. The dot-named objects are hidden, so we need to list them as

\begin{Schunk}
\begin{Sinput}
> ls(pattern = "^\\.", all.names = TRUE)
\end{Sinput}
\begin{Soutput}
 [1] ".ameta"    ".aname"    ".aroot"    ".ffn"      ".imeta"   
 [6] ".iname"    ".inroot"   ".iroot"    ".oroot"    ".outfn"   
[11] ".pfn"      ".pisaPath" ".pmeta"    ".pname"    ".proot"   
[16] ".reproot"  ".rnwfn"    ".smeta"    ".sname"    ".sroot"   
\end{Soutput}
\end{Schunk}


\clearpage
\section{To end}
Reset working directory, if needed.

\begin{Schunk}
\begin{Sinput}
> opts_knit$set(root.dir = oldwd)
> if (interactive()) setwd(oldwd)
\end{Sinput}
\end{Schunk}


%% knitr child handling
%%
%<<child='child_file.Rnw'>>=
%@
%<<result="hide">>=
%out <- ""
%for ( linija in levels(phenodata$Celicna.linija))
%out <- paste(out,knit_child("child_file.rnw",quiet=TRUE))
%@
%out
%%
% ----------------------------------------------------------------
%\bibliographystyle{chicago}
%\addcontentsline{toc}{section}{\refname}
%\bibliography{ab-general}
%--------------------------------------------------------------

%\clearpage
%\appendix
%\phantomsection\addcontentsline{toc}{section}{\appendixname}
%\section{\R\ funkcije}
%\input{}

\clearpage
\section*{SessionInfo}
Windows 10 x64 (build 19042) 
\begin{itemize}\raggedright
  \item R version 4.0.2 (2020-06-22), \verb|x86_64-w64-mingw32|
  \item Locale: \verb|LC_COLLATE=Slovenian_Slovenia.1250|, \verb|LC_CTYPE=Slovenian_Slovenia.1250|, \verb|LC_MONETARY=Slovenian_Slovenia.1250|, \verb|LC_NUMERIC=C|, \verb|LC_TIME=Slovenian_Slovenia.1250|
  \item Running under: \verb|Windows 10 x64 (build 19042)|
  \item Matrix products: default
  \item Base packages: base, datasets, graphics, grDevices,
    methods, stats, utils
  \item Other packages: knitr~1.30, pisar~0.1.0.9000
  \item Loaded via a namespace (and not attached):
    cellranger~1.1.0, compiler~4.0.2, crayon~1.3.4, curl~4.3,
    data.table~1.13.2, ellipsis~0.3.1, evaluate~0.14,
    forcats~0.5.0, foreign~0.8-80, formatR~1.7, haven~2.3.1,
    hms~1.0.0, lifecycle~0.2.0, magrittr~2.0.1,
    openxlsx~4.2.3, pillar~1.4.7, pkgconfig~2.0.3, Rcpp~1.0.5,
    readxl~1.3.1, rio~0.5.16, rlang~0.4.10, stringi~1.5.3,
    stringr~1.4.0, tibble~3.0.4, tools~4.0.2, vctrs~0.3.6,
    xfun~0.19, zip~2.1.1
\end{itemize}


\end{document}
% ----------------------------------------------------------------
